\documentclass[twoside,openany]{book}
\date{}

\usepackage{amssymb,amsmath,amsthm,verbatim,graphicx,bm,thmtools}
\usepackage{xcolor}
\usepackage[utf8]{inputenc}
\usepackage[english]{babel}
\usepackage[babel]{csquotes}
\MakeOuterQuote{"}
\MakeAutoQuote{<}{>}
\usepackage{marvosym}
\usepackage{enumitem}
\usepackage{mathtools}
\usepackage{todonotes}


%%%%%%%%%%%%%%%   Geometrie für Randbreite   %%%%%%%%%%%%%%%%%
\usepackage{geometry}
\geometry{
  left=2cm,
  right=2cm,
  top=2cm,
  bottom=3.2cm,
  bindingoffset=0mm
}
%%%%%%%%%%%%%%%%%%%%%%%%%%%%%%%%%%%%%%%%%%%%%%%%%%%%%%%%%%%%%
\usepackage{fancyhdr}
\pagestyle{fancy}
\fancyhf{}% Clear header/footer

\renewcommand{\headrulewidth}{0pt}%KeineLini

%%%%%%%%%%%%%%%%%%%%%%%%%%%%%%%%%%%%%%%%%%%%%%%%%%%%%%%%%%%%%%

\usepackage{chngcntr}
\counterwithout{footnote}{chapter}

%%%%%%%%%%%%%%%%%%%%%%%%%%%%%%%%%%%%%%%%%%%%%%%%%%%%%%%%%%%%%%

\usepackage{tabularx}
%

\usepackage{tikz}
\usetikzlibrary{patterns}
%\usepgfplotslibrary{fillbetween}

%%%%%%%%%%%%%%%%%%%%%%%%%%%%%%%%%%%%%%%%%%%%%%%%%%%%%%%%%%%%%%%

\numberwithin{equation}{chapter} 

\newcommand{\red}{\textcolor{red}}
\newcommand{\blue}{\textcolor{blue}}
\newcommand{\black}{\textcolor{black}}
\newcommand{\dive}{\mathrm{div}\,}
\newcommand{\eps}{\varepsilon}
\newcommand{\R}{\mathbb{R}}
\newcommand{\N}{\mathbb{N}}
\newcommand{\Z}{\mathbb{Z}}
\newcommand{\Q}{\mathbb{Q}}
\newcommand{\bbr}{\mathbb{R}}
\newcommand{\bbn}{\mathbb{N}}
\newcommand{\IK}{\mathbb{K}}
\newcommand{\ID}{\mathbb{D}}
\newcommand{\sss}{\scriptscriptstyle}
\newcommand{\q}{^}
\renewcommand{\emptyset}{\varnothing}
\renewcommand{\subset}{\subseteq}
\renewcommand{\supset}{\supseteq}
\newcommand{\Pot}{\mathrm{Pot}}
\newcommand{\CalA}{\mathcal{A}}
\newcommand{\CalB}{\mathcal{B}}
\newcommand{\defeq}{\coloneqq}

\definecolor{newgray}{rgb}{0.97,0.97,1.00}

\newcommand{\cbox}[2]{%
  \[\colorbox{#1}{%
      \addtolength{\linewidth}{-2\fboxsep}%
      \addtolength{\linewidth}{-2\fboxrule}%
      \begin{minipage}{\linewidth}% 
      \vspace{-3.5mm}
      \begin{align}#2\end{align}\notag%
      \end{minipage}%
    }\]%
}

\definecolor{newblue}{rgb}{0.0,0.0,0.30}
\newtheoremstyle%
 {theorem}%
 {}{}%
 {\itshape}
 {}%
 {\color{newblue}\bfseries}%
 {:}%
 { }{}

\newcommand{\grid}[1][2]{%
\vspace*{0.1cm}
\noindent
\begin{tikzpicture}
	% Size of the grid
	\def\xmin{0} \def\xmax{17.5} \def\ymin{0} \def\ymax{#1}
	\draw [step=0.5cm,gray, ultra thin] (\xmin,\ymin) grid (\xmax,\ymax);
	\draw [step=1cm,gray, thin] (\xmin,\ymin) grid (\xmax,\ymax);
	\draw [step=5cm,gray,thick] (\xmin,\ymin) grid (\xmax,\ymax);
\end{tikzpicture}
}

\begin{document}

${}$\vspace{-2cm}

\noindent
\begin{minipage}[l]{0.35\textwidth}
\noindent %Katholische Universit\"at Eichst\"att-Ingolstadt\\
%Mathematisch-Geographische Fakult\"at\\
\\ 
Mathematical Institute for\\
Machine Learning and Data Science
\end{minipage}
\begin{minipage}[l]{0.25\textwidth}
\includegraphics[width=\textwidth]{mids_logo.png}
\end{minipage}
\begin{minipage}[r]{0.3\textwidth}
 \hspace{1cm}\includegraphics[width=\textwidth]{KU_Logo}
\end{minipage}

\vspace*{-0.4cm}

\begin{center}
{\LARGE \textbf{The MIDS vocab test / book}}
\end{center}

\noindent
\textbf{What is this about?}

Learning mathematics is similar to learning a language.
If when learning English, you would have to constantly look up the most basic words
(e.g., "I", "you", etc.), you would never be able to have a conversation.
Similarly, if you constantly have to look up the most basic mathematical
concepts and theorems (e.g., the definition for the convergence of a sequence),
you will not be able to get much out of a lecture, or to reasonably solve exercises.

Thus, \emph{you have to know the most important concepts and theorems \textbf{by heart}}.
The following is an attempt to identify these most important things.

It is not enough to \emph{only} know the following facts,
but they provide a basis on which you can further build.

\noindent\rule{\textwidth}{1pt}\\



\begin{center}
{\LARGE \textbf{The absolute basics}}
\end{center}

\noindent
Let $A,B$ be sets.
What does $A \subset B$ mean (definition)?

\grid

\medskip{}

\noindent
When is a function $f : X \to Y$ called \textbf{injective}?

\grid

\medskip{}

\noindent
When is a function $f : X \to Y$ called \textbf{surjective}?

\grid

\medskip{}

\noindent
Let $f : X \to Y$.
Given $M \subset Y$, what is the definition of the \textbf{inverse image} $f^{-1}(M)$?

\grid

\medskip{}

\noindent
How can one express $(A \cup B)^c$ without using the "$\cup$" operation?

\grid

\medskip{}

\noindent
What is the \textbf{negation} of the statement
\[
  \forall \, x \in M : A(x),
\]
where $A(x)$ is some statement depending on $x$?

\grid

\medskip{}

\noindent
How does one prove the statement $A \Rightarrow B$ using
\textbf{contraposition}?

\grid


\medskip{}

\noindent
How is the \textbf{composition} $f \circ g$ of two functions defined?

\grid


%\vspace*{0.1cm}
%\noindent
%\begin{tikzpicture}
%	% Size of the grid
%	\def\xmin{0} \def\xmax{17.5} \def\ymin{0} \def\ymax{2}
%	\draw [step=0.5cm,gray, ultra thin] (\xmin,\ymin) grid (\xmax,\ymax);
%	\draw [step=1cm,gray, thin] (\xmin,\ymin) grid (\xmax,\ymax);
%	\draw [step=5cm,gray,thick] (\xmin,\ymin) grid (\xmax,\ymax);
%\end{tikzpicture}

\noindent\rule{\textwidth}{0.5pt}\\

\begin{center}
{\LARGE \textbf{Analysis I}}
\end{center}

\noindent
What is the relation between $|x + y|$ and $|x| + |y|$ for real numbers $x,y$?

\grid

\medskip{}

\noindent
Let $M \subset \R$.
What is the definition of the \textbf{infimum} and the \textbf{supremum} of $M$?

\grid

\medskip{}

\noindent
Let $(x_n)_{n \in \N}$ be a sequence of real numbers, and let $x \in \R$.
What is the definition of $x_n \xrightarrow[n \to \infty]{} x$?

\grid

\medskip{}

\noindent
If $x_n \to x$ and $y_n \to y$, what can you say about $(x_n + y_n)_{n \in \N}$
and $(x_n \cdot y_n)_{n \in \N}$?

\grid

\medskip{}

\noindent
What does the \textbf{squeeze theorem} (or \textbf{sandwich theorem}) for sequences say?

\grid

\medskip{}

\noindent
When is a sequence $(x_n)_{n \in \N}$ called a \textbf{Cauchy sequence} (definition)?
What can you say about the convergence of a Cauchy sequence?

\grid

\medskip{}

\noindent
Let $(x_n)_{n \in \N}$ be a sequence of real numbers.
What does it mean for the \textbf{series} $\sum_{n=1}^\infty x_n$ to converge?
What does it mean for the \textbf{series} $\sum_{n=1}^\infty x_n$ to converge \emph{absolutely}?

\grid[3]


\medskip{}

\noindent
If the series $\sum_{n=1}^\infty x_n$ converges, what can you say about the convergence
of the sequence $(x_n)_{n \in \N}$?

\grid[3]


\medskip{}

\noindent
What is the \textbf{geometric series}?
What can you say about its convergence and limit?

\grid[3]

\medskip{}

\noindent
What does the \textbf{root test} say?

\grid

\medskip{}

\noindent
What does the \textbf{ratio test} say?

\grid

\medskip{}

\noindent
Let $M \subset \R$ and $f : M \to \R$ a function, and $x_0 \in M$.
What does it mean for $f$ to be \textbf{continuous} at $x_0$ (formal definition)?

\grid


\medskip{}

\noindent
Let $M \subset \R$ and $f : M \to \R$ a function, and $x_0 \in M$.
How can one characterize the continuity of $f$ at $x_0$ using \emph{sequences}?

\grid

\medskip{}

\noindent
What does the \textbf{intermediate value theorem} say?

\grid

\medskip{}

\noindent
What does the \textbf{extreme value theorem} say?

\grid

\medskip{}

\noindent
Let $I \subset \R$ an interval and $f : I \to \R$.
What does it mean for $f$ to be \textbf{increasing/decreasing/strictly increasing}?

\grid

\medskip{}

\noindent
Let $I \subset \R$ an interval and $f : I \to \R$, as well as $x_0 \in I$.
What does it mean for $f$ to be \textbf{differentiable} at $x_0$?
How is the \textbf{derivative} of $f$ at $x_0$ defined?

\grid

\medskip{}

\noindent
What do the \textbf{chain rule}, the \textbf{product rule}, and the \textbf{quotient rule} say?

\grid[4]

\medskip{}

\noindent
What does the \textbf{mean value theorem} say?

\grid

\medskip{}

\noindent
Let $I \subset \R$ an interval and let $f : I \to \R$ be differentiable.
How can one check if $f$ is increasing?

\grid


\medskip{}

\noindent
What are the derivatives of $f(x) = x^n$ and of the functions
$\sin, \cos, \exp$, and $\ln$ (the natural logarithm)?

\grid[3]

\medskip{}

\noindent
Let $I \subset \R$ an interval and $f : I \to \R$, as well as $x_0 \in I$.
What does it mean for $f$ to have a \textbf{local/global maximum} at $x_0$?

\grid[3]

\medskip{}

\noindent
Let $I \subset \R$ an interval and $f : I \to \R$, and let $x_0$ be an \emph{interior point} of $I$.
If $f$ is differentiable at $x_0$ and has a local maximum at $x_0$,
what can you say about $f'(x_0)$?

\grid

\todo[inline]{Add some stuff about integration?!}

\noindent\rule{\textwidth}{0.5pt}\\

\begin{center}
{\LARGE \textbf{Linear Algebra I}}
\end{center}

\todo[inline]{Write this!}

\noindent\rule{\textwidth}{0.5pt}


\end{document}

